%minimal
\documentclass{article}
\begin{document}
Hola
\end{document}




%\documentclass[11pt,a4paper,oneside]{report}
\documentclass[twocolumn]{article}

\usepackage[spanish]{babel}
\usepackage[utf8]{inputenc}
%\usepackage[latin1]{inputenc}
% remove temporal files if you disable spanish options
\usepackage{graphicx}
\usepackage{amsmath}
\usepackage[mathscr]{euscript}
%\usepackage{showframe}
\usepackage{fullpage}

\usepackage{setspace}
\setstretch{1.3}

\begin{document}
\nocite{*}

\title{Descibrimiento de informacion en textos: Tarea 6}
\author{Jordi Colomer}
\maketitle

\tableofcontents
\listoffigures
\listoftables

\begin{abstract}
resumen
\end{abstract}

%\tableofcontents

\section{Introduccion}


\begin{figure}[h!]
\subfloat{\includegraphics[width=.6\textwidth]{/tmp/dist.pdf}}
\caption{Distribution}
\end{figure}

\begin{table}[!htbp]
\centering\begin{tabular}{ lll }
\hline
n & n & n \\
\hline
1 & 2 & 3 \\
4 & 5 & 6 \\
7 & 8 & 9 \\
\hline
\end{tabular}
\caption{Simulation}\label{fig:1}
\end{table}

\newpage
\input{filename.tex}

\begin{itemize}
\item blah
\item bla
\item \textbf{Exactitud}: Numero de instancias clasificadas correctamente.
\end{itemize}

\begin{enumerate}
  \item The first item
  \item The second item
  \item The third etc \ldots
\end{enumerate}

% allowed symbols a-z A-Z 0-9 and -
% no multiple extensions!
% http://wiki.lyx.org/LaTeX/FilesWithSpecialChars

\begin{figure}[!htbp]
\centering\includegraphics[width=140mm]{res/network.png}
\caption{diseno de la red neuronal}
\end{figure}


\begin{figure}[!htbp]
\centering\includegraphics[width=.5\textwidth]{backpropagation_1.pdf}
\caption{backpropagation1}\label{fig:1}
\end{figure}

\ref{fig:backpropagation1}

\cite{Jurafsky}

\begin{thebibliography}{99}
\bibitem{Jurafsky} AUTHORS (2008),\emph{TITLE}.PUBLISHED
\bibitem{Marcu} Marcu, Daniel (2000) \emph{The Theory and Practice of Discourse Parsing and Summarization}.
\bibitem{precision} http://en.wikipedia.org/wiki/Precision\_and\_recall
\end{thebibliography}

\begin{thebibliography}{99}
\bibitem{UNED} Spina, Damiano, et al. "UNED Online Reputation Monitoring Team at RepLab 2013."
\bibitem{REINA} Berrocal, José Luis Alonso, Carlos G. Figuerola, and Ángel Zazo Rodríguez. "REINA at RepLab2013 Topic Detection Task: Community Detection."
\bibitem{Wang} Wang, X., McCallum, A.: Topics over time: a non-markov continuous-time model of topical trends. In: Proceedings of the 12th ACM SIGKDD international conference on Knowledge discovery and data mining. pp. 424{433. KDD '06, ACM, New York, NY, USA (2006)
\bibitem{Spina} Spina, D., Gonzalo, J., Amigo, E.: Discovering filter keywords for company name disambiguation in twitter. Expert Systems with Applications 40(12), 4986 - 5003 (2013)
\end{thebibliography}

%bibtex version,to compile: pdflatex base;bibtex base;pdflatex base;pdflatex base (you must cite them in order to appear)
\bibliographystyle{plain}
\bibliography{papers}

\end{document}



% small table factor reduce shrink
\usepackage{graphicx}
\begin{table}[!htbp]
\centering
\tabcolsep=0.11cm
\scalebox{0.72}{
\begin{tabular}{ llllllll }
\hline
n & success & distance & generations & time (s) & nodes & configuration \\
\hline
\input{table.tex}
\hline
\end{tabular}
}
\caption{Results for different configurations.}\label{fig:t1}
\end{table}

\begin{verbatim}
2389c
\end{verbatim}

%span
\begin{table}[!htbp]
\centering\begin{tabular}{ lllllll }
\hline
Medida & \multicolumn{2}{l}{Componente principal} & \multicolumn{2}{l}{Local Linear Embedding}  & \multicolumn{2}{l}{Mapas difusión} \\
 & RBF & Poly & RBF & Poly & RBF & Poly \\
\hline
Correlation coefficient&                  0.0806 & -0.0209 & 0.0817 & 0.0818 & 0.2355 & 0.1499 \\
Mean absolute error     &              4788.57 & 4874.31 & 4726.43 & 4712.26 & 4786.83 & 4921.65 \\
Root mean squared error  &             8667.81  & 8689.74 & 8636.21 & 8632.52 & 8553.90 & 16510.44 \\
Relative absolute error   &              85.48\% & 87.01\% & 84.37\% & 84.11\% & 85.44\% & 87.85\% \\
Root relative squared error&            107.15\% & 107.42\% & 106.76\% & 106.72\% & 105.74\% & 204.11\% \\
Total Number of Instances   &           831      & 831 & 831 & 831 & 831 & 831 \\
\hline
\end{tabular}
\caption{Simulation}\label{fig:1}
\end{table}

\begin{equation}\label{eq1}
cos^2(x^k, y_p)=\frac{(x^k . y_p)^2}{||x^k||^2||y_p||^2}, k=1..Q
\end{equation}
\eqref{eq1}


\includegraphics[trim=1cm 2cm 3cm 4cm, clip=true, totalheight=0.5\textheight, angle=90]{figure}

%alternative equation and text
\[
 \text{Correctness}(e,e') =
  \begin{cases}
   1 & L(e) = L(e') \leftrightarrow C(e) = C(e') \\
   0       & otherwise
  \end{cases}
\]



%non floating figures
\usepackage{caption}

\begin{center}

\begin{tabular}{ llll }
\hline
Algorithm & Reliability & Sensitivity & F-Measure \\
\hline
ALLINONE & \input{clean/output/getPartitionAllinone.reliability} & \input{clean/output/getPartitionAllinone.sensitivity} & \input{clean/output/getPartitionAllinone.fmeasure} \\
ALLINALL & \input{clean/output/getPartitionAllinall.reliability} & \input{clean/output/getPartitionAllinall.sensitivity} & \input{clean/output/getPartitionAllinall.fmeasure} \\
Jaccard & \input{clean/output/jaccard.reliability} & \input{clean/output/jaccard.sensitivity} & \input{clean/output/jaccard.fmeasure} \\
JaccardW & \input{clean/output/jaccardw.reliability} & \input{clean/output/jaccardw.sensitivity} & \input{clean/output/jaccardw.fmeasure} \\
Cosine & \input{clean/output/cosine.reliability} & \input{clean/output/cosine.sensitivity} & \input{clean/output/cosine.fmeasure} \\
Levenshtein & \input{clean/output/levenshtein.reliability} & \input{clean/output/levenshtein.sensitivity} & \input{clean/output/levenshtein.fmeasure} \\
\hline
\end{tabular}

\captionof{figure}{text}\label{labelname}
%\captionof{table}{text}\label{labelname}
\end{center}

#bold face,negrita,negreta
\textbf{whatevs}
